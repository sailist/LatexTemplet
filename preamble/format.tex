\usepackage[UTF8,fontset=windowsnew,heading=true]{ctex}
\ctexset{
	section = {
	number = 第\chinese{section}章,
	format = \zihao{3}\bfseries,
	},
	subsection = {
	number = \arabic{section}.\arabic{subsection},
	format = \Large\bfseries
	},
	subsubsection = {
	number = \arabic{section}.\arabic{subsection}.\arabic{subsubsection},
	format = \Large\bfseries,
	},
}

\newcommand{\fontpath}{../font/}
% mainfont 一般就是论文中西文部分默认使用的字体
% \setCJKmainfont{SourceHanSerifCN-Regular.otf}[
% Path=\fontpath,
% BoldFont=SourceHanSerifCN-Bold.otf]
% \setmainfont{Microsoft YaHei UI}

% % sans 一般是无衬线字体。可能出现在大标题等显眼的位置
% \setCJKsansfont[Path=\fontpath]{SourceHanSerifCN-Bold.otf}%设置字体族,\textsf{这样就显示微软雅黑}
% \setsansfont[Path=\fontpath]{SourceHanSerifCN-Bold.otf}

% % mono 是默认的等宽字体。一般用于排版程序代码
% \setCJKmonofont[Path=\fontpath]{YaheiConsolasHybrid.ttf}
% \setmonofont[Path=\fontpath]{YaheiConsolasHybrid.ttf}


\graphicspath{{./}{./contents/}{./contents/fig/}}%设置图片可能存在的路径
\newcommand{\figpath}[1]{contents/fig/#1}

% Document
\setlength{\parindent}{2em}%设置首行缩进
\linespread{1}%设置行距
\setlength{\parskip}{0.5em}%设置段间距
\setcounter{tocdepth}{4}%设置目录级数
\setcounter{secnumdepth}{4}