\section{物理层}
\subsection{基本定理}
\subsubsection{傅里叶分析}
19世纪早期,傅里叶证明了任何一个行为合理的周期函数,都可以表示成用正弦函数和余弦函数组成的无穷级数。

\subsubsection{尼奎斯特定理}
用来表示\textbf{有限带宽的无噪声信道}的最大数据传输速率:

\begin{eqbox}
    每秒B次采样下:
    \begin{equation}
        M=2B*log_2(V)  
    \end{equation}
    \tcblower
    B:带宽;V:离散
\end{eqbox}
\subsubsection{香农定理}
用来表示有限带宽的有噪声信道的最大数据传输速率:
\begin{eqbox}
    每秒B次采样下:
    \begin{equation}
        M=B*log_2(2*(1+S/N))  
    \end{equation}
    \tcblower
    B:带宽;S/N:信噪比,即信号功率S和噪声功率N的比值
    \tcblower
    分贝(dB):信噪比的对数表现形式,dB=10*log(10*S/N)
\end{eqbox}

\subsection{其他名词解释}
\subsubsection{信号}
可以简单的认为,信号就是一个频率
\begin{figure}[H]
    \centering
    
    \caption[width=0.8\textwidth]{fig/signal.png}
    \caption{该图表示的信号中,信号只有高低电平两种,因此一个字符(symbol)只能表示一个bit(代表0或1}
    \label{fig:信号}
\end{figure}

书中(P99)提到,利用有限带宽的一种更有效策略是使用两个以上的信号级别 。例如,采用 4 个电压级别,我们可以用单个符号( symbol) 一次携带 2 个比特。只要接收器收到的信号强度足够大到能区分出信号的 4 个级别,这种方案就切实可行。此时信号变化的速率只是比特率的一半,因而减少了所需的带宽。 

\subsubsection{波特率}
信号改变的速率称为符号率(symbol rate),也就是波特率

\subsection{数字调制}
有线和无线信道运载模拟信号,模拟信号可表示成诸如连续变化的电压、光照强度或声音强度。为了发送数字信息,我们必须设法用模拟信号来表示比特。比特与代表它们的信号之间的转换过程称为\textbf{数字调制( digit modulation )}。 
\subsubsection{基带传输}
数字比特直接转换为信号有一些方案,即基带传输,即信号的传输占据传输介质上从零到最大值之间的全部频率,而最大频率则取决于信令速率。这是有线介质普遍使用的一种调制方法 

基带传输的编码方案有:
\begin{parabox}{不归零}
    数字调制的最直接形式是用正电压表示 1 ,用负电压表示 0。对光纤而言,可用光的存
在表示 1 ,没有光表示 0。这种编码方案称为不归零( NRZ, Non-Return-to-Zero )     
\end{parabox}

\begin{parabox}{曼彻斯特编码}

    曼彻斯特编码
\end{parabox}

\begin{parabox}{差分曼彻斯特编码}
    差分曼彻斯特编码
\end{parabox}

注意:两种曼彻斯特编码的1、0位的定义没有严格的要求,可以与上图表示的相反,只要在过程中完全一致即可


\subsubsection{通带传输}
考虑通过调节载波信号的幅值、相位或频率来运载比特的调制模式。这些转换方案导致了通带传输( passband transmission),即信号占据了以载波信号频率为中心的一段频带。这是无线和光纤信道最常使用的调制方法,因为在这样的传输介质中只能在给定的频带中传输信号。 
\subsubsection{多路复用}
信道通常被多个信号共享(以便节约成本),这种信道的共享形式称为多路复用技术( multiplexing ),多路复用技术可以通过几种不同的方式实现。我们将给出一些时分复用、频分复用和码分复用的多路复用方法。 
分类:
\begin{enumerate}
    \item 时分(TDM,time division multiplexing):一条物理信道按时间分成若干个时间片轮流地分配给多个信号使用。
    \item 频分(FDM,frequency division multiplexing):按频谱划分信道,不同频率的信号可以在同一信道内传输。
    \item 波分(WDM,wavelength division multiplexing):频分多路复用的一种,利用光纤信道的巨大带宽,同一光纤可以同时传输一组不同波长的光信号,并且不会互相影响。
\end{enumerate}



