\section{数据链路层}
使用物理层提供的服务在通信信道上发送和接收比特,完成以下功能:
    \begin{enumerate}
        \item 向网络层提供一个良好的服务接口
        \item 处理传输错误
        \item 调节数据流,保证慢速的接收方不会被快速的发送方淹没
        
    \end{enumerate}	
    为了实现上述功能,数据链路层通过从网络层获取数据包,并将之包装成帧的形式。也因此,数据链路层的工作核心就是帧的管理。
    
    \begin{quotebox}
        帧的格式:帧头+有效载荷(存放数据包)+帧尾的帧(frame)    
    \end{quotebox}
    
    \subsection{提供给网络层的服务}
    一般提供以下三种情况的服务:
    \begin{enumerate}
        \item 无确认的无连接服务:源机器向目标机器发送独立的帧,目标机器不对这些帧做确认,不需要建立逻辑连接。适用于错误率低或者实时通信(语音传输)的情况。
        \item 有确认的无连接服务:源机器向目标机器发送独立的帧,目标机器会对这些帧进行确认。不需要建立逻辑连接。适用于不可靠的信道(WIFI)。
        \item 有确认的有连接服务:源机器和目标机器在传输任何一个数据之前要建立一个连接,保证目标机器按照正确的顺序接受一个帧。适用于长距离且不可靠的链路(卫星信道,长途电话)
    \end{enumerate}
    为了\textbf{检错和纠错},数据链路层将比特流拆分成多个离散的帧,为每个帧计算一个称为校验和的短令牌,并将该校验和放在帧中一起传输。

    拆分比特流需要解决的问题:
    \begin{enumerate}
        \item 帧的边界问题:如何识别帧的边界
        \item 帧的透明传输(填充)问题:如果帧的数据中出现和边界一样的flag该如何防止被识别为边界。
    \end{enumerate}
    
    \subsection{数据链路层解决的问题}
    根据上面提到的内容,数据链路层主要需要解决以下几个问题:
    \begin{enumerate}
        \item 成帧:保证易于识别帧的边界,并且不会出现边界误识别(透明传输问题)
        \item 差错控制:保证数据正确性,并确保所有的帧只被递交给目标机器上的网络层一次,并且是按照正确的顺序。
        \item 尽可能的提高传输速度。
    \end{enumerate}

    \begin{quotebox}
        实际上,每一层要解决的问题都可以被概括为提高速度和保证安全性。比如数据链路层上,为了提高速度而研究数据链路层协议,为了保证安全性,提出了帧和帧的差错控制。
    \end{quotebox}

    \subsubsection{成帧}
    \begin{parabox}{字符计数法}
        
    \end{parabox}
    \begin{parabox}{字节填充的标志字节法}
        发送方使用标志字节(FLAG)作为开始和结束;使用转义字节(ESC)表示其后的字节为数据字节而不是标志字节或转义字节(类似java中斜杠或者换行符用"\\"和"\r"表示)。
    \end{parabox}

    \begin{parabox}{比特填充的标志比特法}
        使用“01111110”表示帧的开始和结束,并且在数据中,若遇到5个连续的比特1,就在其后填充一个比特0。接收方除了将首尾的“01111110”删除外,还要将数据中的5个连续比特1后的比特0删除。        
    \end{parabox}

    \begin{parabox}{物理层编码违禁法}
        使用“不会出现在常规数据中”的冗余比特作为边界。
        
        优点:是除了开始和结束的填充外,不再需要填充额外的数据
        \tcblower
        Bit 1:高-低 电平对

        Bit 0:低-高 电平对

        边界:高-高,低-低
    \end{parabox}
    \subsubsection{差错控制}
    \subsubsection{流量控制}
    发送方发送帧的速度超过了接收方能够接收这些帧的速度,而导致丢帧。

    \subsection{基本数据链路层协议}
    在解决问题的时候,首先假设一个最基本的模型,使他能够满足最简单的需求,随后逐步取消限制,增加问题,最后达到理论上的最优。

    \subsubsection{乌托邦式的单工协议}
    \subsubsection{有错信道上的单工停-等式协议}
    \subsubsection{滑动窗口协议}
    \begin{parabox}{捎带确认}
        当到达一个数据帧时,接收方并不是立即发送一个单独的控制帧;而是抑制自己并开始等待,直到网络层传递给它下一个要发送的数据包。然后,确认信息被附加在往外发送的数据帧上(使用帧头的 ack 宇段〉。      
    \end{parabox}
    \begin{parabox}{一位滑动窗口协议}
        
    \end{parabox}
    
    \begin{parabox}{回退N与选择重传}
        站在效率的角度,长发送时间、高带宽和短帧这三者组合在一起就是一种灾难。 
        

    \end{parabox}